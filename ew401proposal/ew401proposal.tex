



EW401 Final Report

BIO
by
Midshipman 1/C Reina Carroll, Midshipman 1/C Jacob Kang,  Midshipman 1/C Ji Lim, Midshipman 1/C Bryan Phan and Midshipman 2/C Levi Hofland
Contact Information: bryanphan168@gmail.com

A Capstone Project Report Submitted to the Faculty of
The Weapons, Robotics and Control Engineering Department
United States Naval Academy, Annapolis, Maryland

Course:  EW401 (Fall) 
Faculty Advisor: Dr. Dennis Evangelista
Department Chair: Prof Brad Bishop
Date: October 2, 2019
 
Contents
1.	Introduction	6
1.1.	Customer Interview	6
1.2.	Additional Background Research	6
2.	Problem Statement	6
2.1.	Problem Statement	6
2.2.	Functions	6
2.3.	Constraints	6
2.4.	Objectives, Pairwise Comparison Chart, and Weightings	6
2.5.	Metrics	7
3.	Related Work	7
4.	Conceptual Designs	7
4.1.	Concept 1:  Title	7
4.2.	Concept 2:  Title	7
4.3.	Concept 3:  Title	7
4.4.	Decision Matrix	7
5.	Ethical Considerations	7
6.	Engineering Standards and Specifications	8
7.	Preliminary Detailed Design	8
7.1.	Component Selection	8
7.2.	Parts List and Budget	8
7.3.	Mechanical Drawings	8
7.4.	Circuit Diagrams	9
7.5.	Prototypes	9
7.6.	Software Structure	9
7.7.	Simulations	10
8.	Proposed Work	10
8.1.	Work Breakdown Structure	10
8.2.	Timeline	10
8.3.	Risk management	10
8.4.	Demonstration and Testing Plan	11
9.	Benchtop Demonstration	11
9.1.	Activities	11
9.2.	Results	11


PREFACE: Using the Template
Duplicate this template file by using the Save As command.  Save the new file as a Word document (.docx) rather than a template (.dotx).  Choose a filename that begins with “EW401_Proposal_XXXX_YYYY” or “EW404_FinalReport_XXXX_YYYY” Where XXXX is a short project title and YYYY is your class year. 
This electronic document1 is a “live” template. The various components of your paper (title, normal or body text, headings, pseudocode etc.) are already defined on the Style Toolbar above and are illustrated throughout the template.  Please do not change font type, size or page layout.   
The Heading Styles help organize the topics on a hierarchical basis and are used to automatically generate the table of contents.  You can force the table to update using right-click Update Field Codes / Update Entire Table. If there are two or more sub-topics, the next level heading should be used (i.e. Styles named “Heading 1”, “Heading 2”, “Heading 3”, and “Heading 4” in the Style Toolbar). Conversely, if there are not at least two sub-topics, then no subheads should be introduced. 
See the Appendix for tips on inserting figures, captions, tables, and equations in MS Word, as well as some general technical writing tips. 
Delete this preface section from your final report! 

Abstract— Your abstract should summarize the objective of your capstone project.  A good guideline is to include a one sentence summary for each section of the report, and dedicate slightly more space to report the results.  Provide specific numerical results or performance measures when possible.  It should be self-contained and as close to 250 words as possible (can be check by highlighting and selecting Review/Word Count. 

    1. Introduction
        1.1. Customer Interview
Describe your customer, their background and how they plan to use the product.   Share highlights from your interview with them.   Include a point of view (POV) statement. 
Dr. Matt Guilliams, a plant systematism and curator of the Santa Barbara Botanical Garden's Clifton Smith Herbarium, studies the flora of California including floristics, biodiversity description, inferring evolutionary patterns, and genetic conservation of rare plants. His recent work includes…..  However, ...His is currently researching the endemic species of flora on the Channel Islands.
Dr. Matt Guilliams wants to identify and collect samples of understudied species of flora on the Channel Islands off the coast of Santa Barbara because he seeks to protect the endemic plants of these islands.
Highlights from the interview are that the species of plant we are trying to collect is located on the Cliffside 2-3 stories tall. While the plant is an unknown species we should work on our system to be able to accurately extract a sample from a woody shrub. Furthermore, the environment surrounding Channel Island may have wind conditions from 15 to 40 miles per hour.
        1.2. Additional Background Research
First, identify and discuss the needs of any additional stakeholders. Second, motivate your problem in terms of the larger potential societal, economic, political, or military impacts of this project.  Make the discussion substantial and support it with non-obvious facts.  Cite and discuss at least 3 additional sources of background research.  Be sure to read and follow the guidance on inserting IEEE-style citations and references in the appendix. 
Innovative methods of identifying and retrieving objects in physically hard to reach places have a wide variety of uses. This technology can be tailored industrial or military use to retrieve or place sensitive hardware. 
For background research we looked at historical air and water temperatures for the environment surrounding the Channel Islands over the past xx years during the month of March, our intended evaluation period.  
. The average air temperature in March has a high of 65F and a low of 49F. The average water temperature has a high of 58.8F and a low of 53.6F. In a case where a human person is thrown off the RHIB and into the water we have 1-2 hours to rescue them before the develop hypothermia in water conditions from 50F-60F. Fog is a common weather feature, especially at San Miguel and Santa Rosa Islands. “Fog is most common in spring and summer, and west of the Santa Cruz Channel. The marine layer fog flows down the coast with the prevailing NW wind, and bends around Point Conception, usually blanketing San Miguel and Santa Rosa, and often the western portion of Santa Cruz Island. Fog frequently is thicker and lingers longer into the day offshore than along the mainland coast. Preliminary data from Santa Cruz Island suggests that geographic variation in the presence and duration of the fog layer has a profound influence on the temperature and humidity regimes.” Furthermore, the island chain itself has 8 major islands with the largest having an area of 96.51 miles. With the specifications of the drones we have in our possession now, Parrot Bebop 2 and the DJI Mavic Pro, the battery life and payload will not be enough to support our mission. In a preliminary design phase we must obtain a system with more power and longer duration. 

https://www.nps.gov/chis/learn/nature/weather.htm
https://www.accuweather.com/en/us/channel-islands-beach-ca/93035/march-weather/2189787 ,
    2. Problem Statement
        2.1. Problem Statement
Present your problem statement.   Be succinct yet as specific as possible.  Include a photo or conceptual diagram (see the Appendix for tips on inserting figures and captions).   Avoid presenting the solution here unless a particular approach reflects a constraint dictated by the customer or legacy considerations.
To build and operate a drone that is capable of surveying, extracting plant samples, and recovering the plant samples securely back to the scientist with minimal interference to the plants Themselves. The extraction device should aim to be able to collect multiple types of plants sought out by our customer, Dr. Matt Guilliams. He wants to identify and collect samples of understudied species of flora on the Channel Islands off the coast of Santa Barbara because he seeks to protect the endemic plants of these islands. 
        2.2. Functions

        2.3. Constraints
Discuss the limits that constrained the design space. These are hard limits on payload, form factor, budget, etc.  Remember, constraints are Boolean design properties that are evaluated on a “pass-fail” basis.  On the other hand, if they are properties you will rate on a continuum they belong in the objectives section.
    • Average Air Temperature in March: High 65 FLow 49 F
    • Average Water Temperature in March: High 58.8 FLow 53.6 F
    • Human Safety in Water: 50-60 F, 1-2 hours
    • Severe Fog/Sea-state will affect operation success
    • Operations will be conducted from a RHIB
    • Operations will be conducted during daylight
    • Cannot damage the environment or other species (possibly endangered)
    • Safety of team during RHIB operations, drone take off, and drone landing
    • Securing sample when returning to RHIB can be difficult with rotor wash and wind
        2.4. Objectives, Pairwise Comparison Chart, and Weightings 
Provide a list of objectives – descriptive adjectives or adverbs.   Consider grouping them hierarchically as an outline or tree.   

For Mobility portion: Part that makes the plant accessible
    • Fly/hover/and extract with stability. Drone must be able to withstand variable  wind speeds while flying and also while hovering. The drone will be able to successfully extract samples if it can fly stably with the payload. 
	For Sampler portion: Part that can actively extract the plant sample
    • To reliably sample plants located on near-vertical cliffs
    • Minimally impact the surrounding environment
    • Properly cut off sample from the rest of the plant (cause minimal harm to the plant)
    • Hold, preserve, and deliver the sample

Using the lowest level objectives, provide a pair-wise comparison chart which illustrates how your customer ranks their importance.   Additionally, provide weightings (scale of 1-10) for each objective.

        2.5. Metrics 
Our team developed four metrics  to accurately evaluate our design. These metrics are Operating Time, Reliability of extraction, Tenacity, and payload. the most important of these metrics is Operating Time since all other metrics depend on the devices ability to successfully operate. Reliability of extraction is the next most important metric since sample extraction is our primary mission.  Tenacity is the third metric due to the environmental factors that must be overcome to accomplish our mission. Lastly our device must have a Payload large enough to support mission necessary equipment

    3. Related Work
Critically discuss and reference at least 3 similar products or research projects.  Include one paragraph per project, critique their solutions, explicitly stating what attribute you tried to emulate and the shortcomings you tried to improve upon.  
End with a paragraph or graphic that compares, contrasts, or synthesizes all the projects you looked at.  
Be sure to read and follow the guidance on inserting IEEE-style citations and references. 
    4. Conceptual Designs
Describe 3 conceptual designs you considered.    For each design provide: a textual description, a concept drawing, and a functional block diagram.  For each metric introduced in the previous section, rate the concept’s potential performance.   Provide engineering justification for each ranking.   Possible approaches would include: rough calculations, identification of possible components and their spec sheets, facts from related work or customer inputs. 
        4.1. Concept 1:  Title

        4.2. Concept 2:  Title

        4.3. Concept 3:  Title

        4.4. Decision Matrix
Include your decision matrix.   Highlight the winning design.   Add a textual justification for selecting it.  
Discuss any additional feedback provided by your customer at the CDR.
    5. Ethical Considerations
Briefly discuss ethical issues that impacted your design approach.  Incorporate the IEEE Code of Ethics or the Professional Engineer’s Code into your discussion when possible. 
    6. Engineering Standards and Specifications
Discuss an engineering standard or specification that is applicable to at least one component of your design.
    7. Preliminary Detailed Design
        7.1. Component Selection
The UAS system we chose to use was the DJI Matrice 100. In respect to our design matrix of “operating time” the Matrice 100 offeres us a performance score of acceptable (2/4) providing sufficient time (20-25 minutes) to complete operation in ideal conditions.  We determined the acceptable operating time of 20-25 minutes from reviewing the DJI Matrice 100 manual, available battery power of the TD48B smart Batteries and estimated overall payload. The DJI Matrice 100 also satisfies another design matrix “Total Required Payload” with a score of acceptable (2/4) being able to carry lightweight mission equipment payload and small plant sample. We determined that the total weight of our payload to include snipper/gripper, attachment system, camera, gimbal and external power should not weigh more than 2-3 lbs. Furthermore the DJI Matrice 100 satisfies another design matrice “tenacity” the ability to operate in windy conditions with a performance score of acceptable (2/4) able to overcome 4-6 m/s which are greater than average wind speed in the Channel Islands in March.  The camera system we chose to use is the Zenmuse X3. We chose the Zenmuse X3 for our beginning stages of prototyping because it provides us with an extremely user accessible gimbal. The controllable range is tilt +30 to -90 degrees, pan plus and minus 320 degrees and roll of plus minus 15 degrees. The camera quality itself is descent with 12 megapixels of resolution and 3.5x zoom. It is able to record at a maximum UDH 4k 25p and has a white balance option for cloudy conditions. Furthermore it has the capacity to store up to 64 gb on a mini SD card. 
For any components not discussed in the previous section, explain how you selected all parts such as actuators, power supplies, computing resources, and sensors include a morph chart.  Provide engineering justifications for your choices.  
        7.2. Parts List and Budget
Provide a parts list and a budget subtotal for those parts. 
Description
Manufacturer
Part Number
In Stock? (Y/N)
Unit Cost
Quantity
Cost
DJI Matrice 100 Kit
DJI
100
N
$3,299
1
$3,299
Samsung Tablet
Samsung


$229.99
1
$229.99
Battery
DJI
TB48D
Y
$239
2
$478
Extra Propellers
DJI
OEM E800 1345
N
$10
4
$40
HERO5 Black
GoPro
---
Y
$180
1
$180
1" Nylon tube
McMaster Carr
8628K68
Y
$9.50
1
$9.50
7/8 Nylon tube
McMaster Carr
8628K64
y
$8.01
1
$8.01
Funnel
McMaster Carr
1479T6
Y
$18.11
1
$18.11
Flexible Magnet
McMaster Carr
5756K32
Y
$2.85
2
$5.70

Provide a complete budget, including labor and overhead using the suggested format.

MATERIALS
Category
Vendor (with POC)
Description of item
Quantity
Unit Price
Cost Estimate

Currently Available Parts
www.dji.com
DJI Matrice 100 Kit
1
$3,299.00
$3,299.00


www.dji.com
Samsung Tablet
1
$229.99
$229.99


www.dji.com
UAV Batteries
2
$239.00
$478.00


www.GoPro.com
HERO5 Black GoPro
1
$180.00
$180.00






$0.00






$0.00











Estimated Shipping Cost



Additional product parts
www.amazon.com
Extra Propellers
4
$10.00
$40.00






$0.00
Project-specific Sub-Total with Shipping Estimate





$3,528.99
Complete Materials Sub-total with Shipping Estimate





$3,568.99
LABOR
Category


Hours
Hourly rate
Cost

Midshipman


520
$25
$13,000

Faculty


64
$60
$3,840

Staff


30
$40
$1,200
Sub-total





$18,040
OVERHEAD
Category


Base Labor Cost
Overhead Rate
Cost

Fringe Benefits


$18,040
35%
$6,314

Facilities


$18,040
50%
$9,020

General Services


$18,040
15%
$2,706
Sub-total





$18,040
TOTAL COST





$39,649
OUT-OF-POCKET COST





$3,529

        7.3. Mechanical Drawings
Include dimensioned drawings of parts you plan on building or having the shop fabricate.   Neat hand drawings are acceptable. 



Figure 1: This is an example of a dimensioned mechanical drawing (left) and an annotated photo (right).   Note this caption was created using References/Insert Caption, which allow MS Word to automatically number and cross reference your figures.

        7.4. Circuit Diagrams
Provide circuit diagrams of any electronics subsystems you will build.

Figure 2  This is an example of an electrical circuit diagram (left), note the component values are included; and an annotated picture of a fabricated circuit (right).  

        7.5. Prototypes
Show pictures of any prototypes of scale models you made.
        7.6. Software Structure 
Your algorithm or controller design should be in the format of pseudo-code (see Figure 3). It is not necessary to include computer programs in their entirety, though interesting excerpts may be included in the text. Appropriately comment any included code. The documentation needs to clearly indicate what code was written by the group members versus code provided by another source (e.g. faculty, TSD, commercial or open-source software, etc.).  For larger, hierarchical programs, represent the functional organization with a flow chart or a block diagram.  For programs utilizing GUIs, screen captures of the user interface are useful.   

Figure 3  Pseudo-code is one of the most compact ways to present computer programs.   Note that a “pseudo-code style” has been defined in the Style Toolbar of the template.  

        7.7. Simulations
This section could include the development of a mathematical model and/or the use of a model to predict behavior and system performance.  If you used simulation or other computer-aided design tools discuss them here. 

Figure 5  This is the response of the feedback control law.   Note that it is scaled appropriately to see the interesting features, and annotated.   When importaing Matlab figures, you may need to increase the font size and line width using Matlab to improve legibility.

    8. Proposed Work
        8.1. Work Breakdown Structure
For capstone work involving more than one student, describe the distribution of the workload over the Spring semester. Discuss any specialization or leadership roles in different disciplines, skills, or subsystems. Note that all group members contribute to the writing of the final report and presentation – it should not be one person’s job alone.  
        8.2. Timeline
Present a proposed Gantt chart, set of milestones or timeline for the Spring semester.   Be sure the font is legible. 
        8.3. Risk management
Discuss the major risks you face, the likelihood they occur and the steps you are taking to mitigate. 
        8.4. Demonstration and Testing Plan
Discuss your approach to testing your project, broken down by sub-system.
Explain in detail how the metrics will be evaluated. 
    9. Benchtop Demonstration
        9.1. Activities
Describe the activities you engaged in during the final weeks of the semester.
        9.2. Results 
Present the results of your benchtop testing. 

References
Instructions on how to insert references in MS Word appear in the Appendix.  The following examples show how most common materials should be referenced using the IEEE standard. Note: parenthetical descriptions should not be included in your reference entries.
    [1] G. O. Young, “Synthetic structure of industrial plastics (Book style with paper title and editor),” in Plastics, 2nd ed. vol. 3, J. Peters, Ed.  New York: McGraw-Hill, 1964, pp. 15–64.
    [2] W.-K. Chen, Linear Networks and Systems (Book style). Belmont, CA: Wadsworth, 1993, pp. 123–135.
    [3] H. Poor, An Introduction to Signal Detection and Estimation (Book style with chapter ref)   New York: Springer-Verlag, 1985, ch. 4.
    [4] B. Smith, “An approach to graphs of linear forms (Unpublished work style),” unpublished.
    [5] C. J. Kaufman, Rocky Mountain Research Lab., Boulder, CO, private communication, May 1995.
    [6] J. U. Duncombe, “Infrared navigation—Part I: An assessment of feasibility (Periodical style),” IEEE Trans. Electron Devices, vol. ED-11, pp. 34–39, Jan. 1959.
    [7] S. P. Bingulac, “On the compatibility of adaptive controllers (Published Conference Proceedings style),” in Proc. 4th Annu. Allerton Conf. Circuits and Systems Theory, New York, 1994, pp. 8–16.
    [8] G. W. Juette and L. E. Zeffanella, “Radio noise currents n short sections on bundle conductors (Presented Conference Paper style),” presented at the IEEE Summer power Meeting, Dallas, TX, Jun. 22–27, 1990, Paper 90 SM 690-0 PWRS.
    [9] J. Williams, “Narrow-band analyzer (Thesis or Dissertation style),” Ph.D. dissertation, Dept. Elect. Eng., Harvard Univ., Cambridge, MA, 1993. 
    [10] N. Kawasaki, “Parametric study of thermal and chemical nonequilibrium nozzle flow,” M.S. thesis, Dept. Electron. Eng., Osaka Univ., Osaka, Japan, 1993.
    [11] J. P. Wilkinson, “Nonlinear resonant circuit devices (Patent style),” U.S. Patent 3 624 12, July 16, 1990. 
    [12] IEEE Criteria for Class IE Electric Systems (Standards style), IEEE Standard 308, 1969.
    [13] R. E. Haskell and C. T. Case, “Transient signal propagation in lossless isotropic plasmas (Report style),” USAF Cambridge Res. Lab., Cambridge, MA Rep. ARCRL-66-234 (II), 1994, vol. 2.
    [14] (Handbook style) Transmission Systems for Communications, 3rd ed., Western Electric Co., Winston-Salem, NC, 1985, pp. 44–60.
    [15]  (Basic Book/Monograph Online Sources) J. K. Author. (year, month, day). Title (edition) [Type of medium]. Volume (issue).	 Available: http://www.(URL)
    [16] J. Jones. (1991, May 10). Networks (2nd ed.) [Online]. Available: http://www.atm.com
    [17] (Journal Online Sources style) K. Author. (year, month). Title. Journal [Type of medium]. Volume(issue), paging if given.	  Available: http://www.(URL)
    [18] R. J. Vidmar. (1992, August). On the use of atmospheric plasmas as electromagnetic reflectors. IEEE Trans. Plasma Sci. [Online]. 21(3). pp. 876–880.   Available: http://www.halcyon.com/pub/journals/21ps03-vidmar


Appendix:   Formatting and Style Guidelines
Appendices are optional and should appear at the end. You may have several appendices if necessary, to include computer programs, large data sets, or detailed drawings. Be sure to note the existence of any appendices or enclosures within the relevant portion of the report text.  
This appendix provides stylistic guidelines and MSWord tips on creating linked and cross referenced documents.  
Abbreviations and Acronyms
Define abbreviations and acronyms the first time they are used in the text, even after they have been defined in the abstract. Abbreviations such as IEEE, SI, MKS, CGS, sc, dc, and rms do not have to be defined. Do not use abbreviations in the title or heads unless they are unavoidable.
Units
The use of SI units is encouraged. English units may be used as secondary units (in parentheses). An exception would be the use of English units as identifiers in trade, such as “3.5-inch disk drive”. Use a zero before decimal points: “0.25”, not “.25”. Use “cm3”, not “cc”. 
Equations
Number equations consecutively. Equation numbers, within parentheses, are to position flush right, as in (2). Punctuate equations with commas or periods when they are part of a sentence, as in
		α  + β  = χ.					        (2)
Note that the equation is centered. Be sure that the symbols in your equation have been defined before or immediately following the equation. Use “(2)”, not “Eq. (2)” or “equation (2)”, except at the beginning of a sentence: “Equation (2) is . . .”
Figures and Tables
Place figures and tables at the top and bottom of pages. Avoid placing them in the middle of pages. Figure captions should be below the figures as shown in Figure 6; table captions should appear above the tables as shown Table 1. To insert captions select from the MS Word toolbar References/Insert Caption.  Reference the figure in the text by choosing the Insert tab and selecting Cross-reference from the Links sub-tab.  Choose the reference type that you want and be sure to select the appropriate entry in the Insert reference to pull-down list.  This will only work if you have used Insert Caption with the figure or table. This can be achieved by right-clicking on the item or from the References Menu.

Table 1 Captioned Table
Table Head
Table Column Head

Table column subhead
Subhead
Subhead
copy
More table copya




Figure 6 Presentation of the Marsh Award, which is given each year to the best 1/C  Robotics and Control Engineering project in memory of ENS David R.  Marsh, USNA Class of 1987, in honor of his enthusiasm for his 1/C project, a voice activated robot.
Axis Labels: Use 8 point Times New Roman for axis label or titles. Use words rather than symbols or abbreviations to avoid confusing the reader. As an example, write the quantity “Magnetization”, or “Magnetization, M”, not just “M”. If including units in the label, present them within parentheses. Do not label axes only with units. In the example, write “Magnetization (A/m)” or “Magnetization {A[m(1)]}”, not just “A/m”. 

References
Note that, while Word includes many bibliography-building tools, one of the easiest methods to build a lengthy reference list is to move to the end of your document and start a new list of references formatted as follows:
1)  Click on the HOME tab at the top of the Word window
2)  In the Paragraph sub-window, click on the down arrow next to the numbered list layout.


3)  Click on the format that shows [#], OR, if that format is not available, select “Define New Number Format” from the bottom of the pop-up window.
4)  If you had to define a new number format, select the “Number style” as “1, 2, 3, …” and change the “Number format” to [1] by typing the [ and the ] around the 1, and removing any other punctuation, as shown at right.  Select OK when done.
5)  Type entries onto the list.  It will automatically update.
6)  To cross-reference the list, use the method described above in the section on Figures and Tables.
The reference/bibliography standard used in our discipline is published by IEEE, a professional engineering society dedicated to control systems, robotics, electronics, etc.  The following examples are provided in the reference section.  Note: parenthetical descriptions do not need to be included in your reference entries. 


